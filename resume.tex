\documentclass{resume}
\usepackage{zh_CN-Adobefonts_external} 
\usepackage{linespacing_fix}
\usepackage{cite}
\usepackage{hyperref}
\hypersetup{
    colorlinks=true,
    linkcolor=cyan,
    filecolor=magenta,      
    urlcolor=blue,
}

\begin{document}
\pagenumbering{gobble}

%***"%"后面的所有内容是注释而非代码,不会输出到最后的PDF中
%***使用本模板,只需要参照输出的PDF,在本文档的相应位置做简单替换即可
%***修改之后,输出更新后的PDF,只需要点击Overleaf中的“Recompile”按钮即可

%在大括号内填写其他信息,最多填写4个,但是如果选择不填信息,
%那么大括号必须空着不写,而不能删除大括号。
%\otherInfo后面的四个大括号里的所有信息都会在一行输出
%如果想要写两行,那就用两次这个指令(\otherInfo{}{}{}{})即可


%***********个人信息**************
\MyName{何致力}
\sepspace
\SimpleEntry{HEZH0014@e.ntu.edu.sg}
\SimpleEntry{+8615982296295}
\SimpleEntry{\href{https://github.com/Shr1mpTop}{个人主页:https://github.com/Shr1mpTop}}

%************照片**************
%照片需要放到images文件夹下,名字必须是you.jpg,注意.jpg后缀(可以去resume.cls第101行处修改),如果不需要照片可以不添加此行命令
%0.15的意思是,照片的宽度是页面宽度的0.15倍,调整大小,避免遮挡文字
\yourphoto{0.14}

%***********教育背景**************
\section{教育背景}
%***第一个大括号里的内容向左对齐,第二个大括号里的内容向右对齐
%***\textbf{}括号里的字是粗体,\textit{}括号里的字是斜体
\datedsubsection{\textbf{南洋理工大学},区块链,\textit{硕士}}{2025.8 - 2026.6 (Pending)}
\datedsubsection{\textbf{电子科技大学},数字媒体技术,\textit{本科}}{2021.9 - 2025.7}
\begin{itemize}
  \item 标兵奖学金 * 2、校社会优秀实践个人 * 1
\end{itemize}

%***********过往经历**************

\section{项目经历}
\datedsubsection{\textbf{基于深度学习胃肠道内窥镜图像智能诊断}}{2024.09 -- 2025.06}
\Content
{面向胃肠道内窥镜图像诊断任务,设计并实现基于 Mamba 架构的多分支视觉模型。}
{负责模型的整体搭建与优化,包括数据预处理与增强、训练与验证流程设计、超参数调优及可视化分析;实现端到端训练与推理,并开发诊断系统原型。}
{在 Kvasir 数据集测试集上取得准确率 87.25\%,优于多种主流对比模型;完成可{\href{https://github.com/Shr1mpTop/Gastrointestinal-Diagnosis-System}{交互诊断系统}},实现在线推理与可视化结果展示。}

\datedsubsection{\textbf{成都晓多科技有限公司},研发实习生}{2024.12 -- 2025.3}
\Content
{参与电商智能客服大语言模型开发,涵盖数据清洗、模型微调和部署。}
{独立搭建模型研发流程,基于DataForce实现训练数据自动采集和清洗,提升数据处理效率35\%。负责针对方太、美的等客户的模型微调与测试,保证对话准确稳定。}
{成功上线30个智能客服Agent,对话准确率超过90\%,显著提升客户服务自动化,助力实现255万人民币回款。}

\datedsubsection{\textbf{《Build and Defense》FPS 塔防游戏},核心功能开发}{2024.03 -- 2024.06}
\Content
{在 Unity 引擎环境下主导建筑系统与 UI 模块的开发与优化,确保游戏核心机制的稳定性与可扩展性。}
{配置实现建筑数据管理与逻辑控制,完成建筑系统的集成;优化 UI 交互流程与性能,提升操作响应速度;设计并实现多种建造元素与塔防策略的逻辑规则}
{推动团队协作,成功交付系统,显著提升游戏策略深度与用户体验。}

\datedsubsection{\textbf{《Time Traveler》3D 解谜游戏},UI 系统设计与实现}{2023.09 -- 2024.01}
\Content
{在游戏设计课程团队项目中,负责 UI 设计与核心功能实现,包括小地图与剧情对话框模块。}
{基于 Unity 引擎完成 UI 布局、交互逻辑与动画效果设计;在小地图模块中实现精确定位与动态更新,帮助玩家快速理解关卡布局;在剧情对话框模块中优化界面美观性与易用性,确保对话信息清晰传达。}
{在课程评审中因 UI 系统功能完善与体验流畅获得导师高度评价,最终项目评分位列全班前 10\%。}

\datedsubsection{\textbf{南洋理工大学人工智能实验室}}{2023.7}
\Content
{参与新冠肺炎CT图像智能诊断项目,负责深度学习模型设计与搭建。}
{完成对照实验和消融实验,撰写并发表国际会议论文。}
{基于自适应融合算法,模型准确率约93\%。}

\section{专业技能}
\datedsubsection{\textbf{编程语言}}{}
\begin{itemize}[leftmargin=2em]
  \item \textbf{Python(精通)}:独立完成深度学习模型构建与优化,熟悉数据处理与计算机视觉技术。
  \item \textbf{C / C++(熟练)}:掌握系统编程、数据结构与算法实现。
  \item \textbf{C\#(熟练)}:参与两款 Unity 游戏开发,负责核心逻辑与功能模块实现。
  \item \textbf{Go(基础)}:实习期间搭建接口服务,熟悉并发模型与接口设计。
\end{itemize}
\datedsubsection{\textbf{工具与框架}}{}
\begin{itemize}[leftmargin=2em]
  \item \textbf{深度学习框架}:熟练使用 PyTorch 与 TensorFlow 完成图像分类、目标检测及自然语言处理任务,具备模型构建、训练优化与部署经验。
  \item \textbf{数据分析工具}:掌握 NumPy 与 Pandas,能够进行高效的数据清洗、特征工程及统计分析。
  \item \textbf{数据库与后端}:具备 SQL 数据库设计与查询优化能力,熟悉 API 设计及后端接口开发流程。
  \item \textbf{游戏与前端开发}:熟练使用 Unity 进行 2D/3D 游戏开发,掌握 HTML5、CSS、JavaScript 进行交互式 Web 页面实现。
\end{itemize}

\sepspace

\section{奖励荣誉}
\begin{itemize}
  \item 标兵奖学金(2024)
  \item 全国大学生市场调查与分析大赛(全国三等奖、省级一等奖,2024)
  \item 标兵奖学金(2023)
  \item 校社会优秀实践个人(2022)
  \item 校园艺术节三等奖(2022)
\end{itemize}
\datedsubsection{\textbf{论文方面}:\href {https://www.ewadirect.com/proceedings/ace/article/view/15239xxx}{He, Z. (2024). Pneumonia image classification using convolutional neural network. \textit{Applied and Computational Engineering}, 67, 255-266.}}{2024}
\end{document}